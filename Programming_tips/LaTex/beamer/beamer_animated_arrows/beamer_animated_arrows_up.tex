\documentclass[compress]{beamer}
\usetheme{Warsaw}
\usepackage{amsmath}
\usepackage{amssymb}
\usepackage{tikz}
\usetikzlibrary{arrows,shapes}
\tikzstyle{every picture}+=[remember picture]
\tikzstyle{na} = [baseline=-.5ex]

\begin{document}

\begin{frame}
\frametitle{Using Arrows in tikz}
  % First page is the equation without arrow
  \begin{itemize}
    \item<2-> Overall mean \tikz[na] \node[coordinate] (s1) {};
    \item<1->[]{%
    \begin{equation}
      y_{ijk} = \tikz[baseline]{ \node[fill=blue!20,anchor=base,rounded corners=2pt]
              (d1) {$\mu$}; }
              + \epsilon_{ijk}
    \end{equation}}%
  \end{itemize}

  % This part is for arrow overlay equation in next page.
  \begin{tikzpicture}[overlay]
  \path<2->[->] (s1) edge [bend left] (d1);
  \end{tikzpicture}
\end{frame}

% End of Document
\end{document}
