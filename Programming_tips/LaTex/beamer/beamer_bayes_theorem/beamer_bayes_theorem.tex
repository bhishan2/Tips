\documentclass{beamer}
\usetheme{Madrid}
\usepackage{tikz}

\newcommand\tikzmark[1]{
  \tikz[remember picture,overlay] \coordinate (#1);
}

\begin{document}

\begin{frame}
\begin{exampleblock}{Baye's theorem}
\[
\tikzmark{ptd}p(\theta\, |\, D) = \frac{\tikzmark{pdt}p(D\,|\,\theta) p(\theta)\tikzmark{pt}}{\tikzmark{pd}p(D)}
\]
\begin{tikzpicture}[
  remember picture,
  overlay,
  expl/.style={draw=orange,fill=orange!30,rounded corners,text width=3cm},
  arrow/.style={red!80!black,ultra thick,->,>=latex}
]
\node[expl]
  (ptdex)
  at (2,-2cm)
  {Some explanation};
\node[expl]
  (pdtex)
  at (4,3.5cm)
  {Some other explanation; this one is a little longer};
\node[expl]
  (pdex)
  at (9,-3cm)
  {Some other explanation};
\node[expl]
  (ptex)
  at (8,3.5cm)
  {Some other explanation; this one is a little longer};
\draw[arrow]
  (ptdex) to[out=100,in=180] ([yshift=0.5ex]{ptd});
\draw[arrow]
  (pdtex.west) to[out=180,in=180] ([yshift=0.5ex]{pdt});
\draw[arrow]
  (pdex.north) to[out=90,in=180] ([yshift=0.5ex]{pd});
\draw[arrow]
  (ptex.east) to[out=0,in=0] ([yshift=0.5ex]{pt});
\end{tikzpicture}
\end{exampleblock}
\end{frame}

\end{document}
