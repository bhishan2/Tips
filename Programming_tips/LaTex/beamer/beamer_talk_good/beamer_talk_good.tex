%!TEX
% Author: Bhishan Poudel
% Date: Aug 21, 2017 Mon
% Topic: Presentation using beamer
%%%%%%%%%%%%%%%%%%%%%%%%%%%%%%%%%%%%%%%%%%%%%%%%%
\documentclass{beamer}
\usepackage[utf8]{inputenc}
\usepackage{utopia} %font utopia imported
\usetheme{Madrid}
\usecolortheme{default}
%Tikz Package for drawing arrows etc.
\usepackage{tikz}
\usetikzlibrary{arrows,shapes}
\tikzstyle{every picture}+=[remember picture]
\tikzstyle{na} = [baseline=-.5ex]
%other packages
\usepackage{amsmath,amssymb}
\usepackage{xcolor}
\usepackage{fancybox}
\usepackage{hyperref}
\hypersetup{colorlinks=true,linkcolor=orange,filecolor=magenta,urlcolor=cyan,
            citecolor=DarkRed, bookmarks=true }
%\highlight[color_name]{Text or math term}
\newcommand{\abs}[1]{\lvert#1\rvert}
\newcommand{\Abs}[1]{\left\lvert#1\right\rvert}

% \highlight[<colour>]{<stuff>}
\newcommand{\highlight}[2][yellow]{\mathchoice
  {\colorbox{#1}{$\displaystyle#2$}}
  {\colorbox{#1}{$\textstyle#2$}}
  {\colorbox{#1}{$\scriptstyle#2$}}
  {\colorbox{#1}{$\scriptscriptstyle#2$}}}

%------------------------------------------------------------
%% Title Page
\title[Astroseminar Fall 2017] %optional
{Title of  Presentation}
% \subtitle{Subtitle}
\author[Poudel, Bhishan] % (optional)
{~B.~Poudel\inst{1} }

\institute[Ohio University] % (optional)
{\inst{1}%
  Department of Physics and Astronomy\\
  Ohio University
}
\date[Aug 21, 2017] % (optional)
{} % We can write text here
\logo{\includegraphics[height=1.5cm]{ou.jpeg}}


%------------------------------------------------------------
% Table of Contents
\AtBeginSection[]
{
  \begin{frame}
    \frametitle{Table of Contents}
    \tableofcontents[currentsection]
  \end{frame}
}


%------------------------------------------------------------
%%%%%%%% Begin of Document%%%%%%%%%%%
\begin{document}
\frame{\titlepage}
\begin{frame}
\frametitle{Table of Contents}
\tableofcontents
\end{frame}
%---------------------------------------------------------


\section{First section}
%---------------------------------------------------------
%Example to make text visible
\begin{frame}
  \frametitle{Example to make text visible using \\item command}
    This is a text in second frame. For the sake of showing an example.

  \begin{itemize}
    \item<1-> Text visible on slide 1
    \item<2-> Text visible on slide 2
    \item<3> Text visible on slides 3
    \item<4-> Text visible on slide 4
  \end{itemize}
\end{frame}


%---------------------------------------------------------
%Example of the \pause command
\begin{frame}
  \frametitle{Example to use pause command}
    In this slide \pause

    the text will be partially visible \pause

    And finally everything will be there
\end{frame}
%---------------------------------------------------------


%
%
%#******************************************************************************
%#==============================================================================
%#          Section : Section2: Highlight
%#==============================================================================
%#******************************************************************************
%
%
\section{Section 2: Highlighted Red Green Blue boxes}\label{sec:sec2}
\begin{frame}
\frametitle{Sample frame title}
  In this slide, some important text will be

  \alert{highlighted} because it's important.
    Please, don't abuse it.

  \begin{block}{Remark}
    Sample text
  \end{block}

  \begin{alertblock}{Important theorem}
    Sample text in red box
  \end{alertblock}
\end{frame}

\begin{frame}
  \begin{alertblock}{Important theorem}
    Sample text in red box
  \end{alertblock}

  \begin{examples}
    Sample text in green box. "Examples" is fixed as block title.
  \end{examples}
\end{frame}
%---------------------------------------------------------


%---------------------------------------------------------
%Two columns
\begin{frame}
\frametitle{Two-column slide}
    \begin{columns}
      \column{0.5\textwidth}
      This is a text in first column.
      $$E=mc^2$$

    \begin{itemize}
      \item First item
      \item Second item
    \end{itemize}

    \column{0.5\textwidth}
    This text will be in the second column
    and on a second tought this is a nice looking
    layout in some cases.
  \end{columns}
\end{frame}
%---------------------------------------------------------


%
%
%#******************************************************************************
%#==============================================================================
%#          Section : Section3: Highlight equation
%#==============================================================================
%#******************************************************************************
%
%
\section{Section 3: Highlight equation using xcolor colorbox}\label{sec:sec3}
\begin{frame}
  \frametitle{Example to highlight equation using xcolor fancybox}


    \begin{align}\label{eq:eq1}
       \colorbox{blue!20}{$ x^2 + y^2$} &= 4
    \end{align}

\end{frame}

\begin{frame}
  \frametitle{highlighe eqn}
  \begin{align}
    \colorbox{blue!20}{$ x^2 + y^2$} &= 9
    \end{align}
    We can also cite another equation (\ref{eq:eq1})
\end{frame}


\begin{examples}
  "Examples" is fixed as block title.
\end{examples}


%
%
%#******************************************************************************
%#==============================================================================
%#          Section 4 : Background hightlight equation terms
%#==============================================================================
%#******************************************************************************
%
%
\section{Section 4: Highlight displaymath term using newcommand}\label{sec:high_eqn}
Example of highlight:
\begin{gather*}
    \highlight[green]{\Abs{\frac{f(x) - f(x_0)}{x-x_0} - \ell}} < \epsilon
\end{gather*}

%
%
%#******************************************************************************
%#==============================================================================
%#          Section : Section 5: Using arrows in tikz
%#==============================================================================
%#******************************************************************************
%
%
\section{Section 5: Tikz Arrows}\label{sec:arrow}
\begin{frame}
\frametitle{Using Arrows in tikz}
  % First page is the equation without arrow
  \begin{itemize}
    \item<2-> Overall mean \tikz[na] \node[coordinate] (s1) {};
    \item<1->[]{%
    \begin{equation}
      y_{ijk} = \tikz[baseline]{ \node[fill=blue!20,anchor=base,rounded corners=2pt]
              (d1) {$\mu$}; }
              + \epsilon_{ijk}
    \end{equation}}%
  \end{itemize}

  % This part is for arrow overlay equation in next page.
  \begin{tikzpicture}[overlay]
  \path<2->[->] (s1) edge [bend left] (d1);
  \end{tikzpicture}

\end{frame}




\end{document}
