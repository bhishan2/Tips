\documentclass[report]{nlctdoc}

\usepackage[a4paper,left=1.75in]{geometry}
\usepackage{alltt}
\usepackage{mfirstuc}
\usepackage{pifont}
\usepackage[utf8]{inputenc}
\ifpdf
 \usepackage[T1]{fontenc}
 \usepackage{metalogo}
\else
 \providecommand{\XeLaTeX}{XeLaTeX}
 \providecommand{\XeTeX}{XeTeX}
\fi
\usepackage{cmap}
\usepackage[colorlinks,
            bookmarks,
            hyperindex=false,
            pdfauthor={Nicola L.C. Talbot},
            pdftitle={mfirstuc.sty: uppercasing first letter},
            pdfkeywords={LaTeX,package,uppercase}]{hyperref}

\IndexPrologue{%
\chapter*{\indexname}\markright{\indexname}
\addcontentsline{toc}{chapter}{\indexname}}

\begin{document}
\MakeShortVerb{|}
 \title{mfirstuc.sty v2.04: 
uppercasing first letter}
 \author{Nicola L.C. Talbot\\[10pt]
Dickimaw Books\\
\url{http://www.dickimaw-books.com/}}

 \date{2016-07-31}
 \maketitle
 \tableofcontents

 \chapter{Introduction}
 \label{sec:intro}

The \styfmt{mfirstuc} package was originally part of the
\styfmt{glossaries} bundle for use with commands like \cs{Gls}, but
as the commands provided by \styfmt{mfirstuc} may be used without
\styfmt{glossaries}, the two have been split into separately
maintained packages.

\begin{important}
The commands described here all have limitations. To minimise
problems, use text-block style semantic commands with one argument
(the text that requires case-changing), and avoid scoped 
declarations.
\end{important}

Here are some examples of semantic commands:
\begin{enumerate}
\item Quoted material:
\begin{verbatim}
\newcommand{\qt}[1]{``#1''}
\end{verbatim}
(or use the \styfmt{csquotes} package). With this, the following
works:
\begin{verbatim}
\makefirstuc{\qt{word}}
\end{verbatim}
This produces:
\begin{display}
\makefirstuc{\qt{word}}
\end{display}
Whereas 
\begin{verbatim}
\makefirstuc{``word''}
\end{verbatim}
fails (no case-change and double open quote becomes two single open
quotes):
\begin{display}
\makefirstuc{``word''}
\end{display}

\item Font styles or colours:
\begin{verbatim}
\newcommand*{\alert}[1]{\textcolor{red}{#1}}
\end{verbatim}
Now the following is possible:
\begin{verbatim}
\makefirstuc{\alert{word}}
\end{verbatim}
This produces
\begin{display}
\newcommand*{\alert}[1]{\textcolor{red}{#1}}%
\makefirstuc{\alert{word}}
\end{display}
Note that \verb|\makefirstuc{\textcolor{red}{word}}| fails
(with an error) because the case-changing interferes with the label.
\end{enumerate}

Define these semantic commands robustly if you intend using
any of the commands that fully expand their argument
(\cs{emakefirstuc}, \cs{ecapitalisewords} and
\cs{ecapitalisefmtwords}).

\chapter{Capitalising the First Letter of a Word}
\label{sec:makefirstuc}

A simple word can be capitalised just using the standard \LaTeX\
upper casing command. For example,
\begin{verbatim}
\MakeUppercase word
\end{verbatim}
but for commands like \cs{Gls} the word may be embedded within the
argument of another command, such as a~font changing command. This
makes things more complicated for a~general purpose solution, so
the \styfmt{mfirstuc} package provides:
\begin{definition}[\DescribeMacro{\makefirstuc}]
\cs{makefirstuc}\marg{stuff}
\end{definition}
This makes the first object of
\meta{stuff} upper case unless \meta{stuff} starts with a control
sequence followed by a non-empty group, in which case the first
object in the group is converted to upper case.
\textbf{No expansion is performed on the argument.}

\begin{important}
If \meta{stuff} starts with a control sequence that takes more than
one argument, the case-changing will always be applied to the
\emph{first} argument. If the text that requires the case change is
in one of the other arguments, you must hide the earlier arguments
in a wrapper command. For example, instead of \verb|\textcolor{red}{text}|
you need to define, say, \verb|\red| as \verb|\color{red}| and use
\verb|\red{text}|.
\end{important}

Examples:
\begin{itemize}
\item |\makefirstuc{abc}| produces \makefirstuc{abc}.

\item |\makefirstuc{\emph{abc}}| produces \makefirstuc{\emph{abc}}
(\ics{MakeUppercase} has been applied to the letter \qt{a} rather
than \cs{emph}). Note however that
\begin{verbatim}
\makefirstuc{{\em abc}}
\end{verbatim}
produces \makefirstuc{{\em abc}} (first object is |{\em abc}| so
this is equivalent to |\MakeUppercase{\em abc}|), and
\begin{verbatim}
{\makefirstuc{\em abc}}
\end{verbatim}
produces {\makefirstuc{\em abc}} (|\em| doesn't have an argument
therefore first object is |\em| and so is equivalent to
|{\MakeUppercase{\em}abc}|).

\item |\makefirstuc{{\'a}bc}| produces \makefirstuc{{\'a}bc}.

\item |\makefirstuc{\ae bc}| produces \makefirstuc{\ae bc}.

\item |\makefirstuc{{\ae}bc}| produces \makefirstuc{{\ae}bc}.

\item |\makefirstuc{{ä}bc}| produces \makefirstuc{{ä}bc}.

\end{itemize}

Note that non-Latin or accented characters appearing at the
start of the text should be placed in a group (even if you are
using the \sty{inputenc} package). The reason for this restriction
is detailed in \sectionref{sec:utf8}.

\textbf{New to version 2.04:} There is now limited support for
UTF-8 characters with the \sty{inputenc} package, provided that
you load \sty{datatool-base} (at least v2.24) before
\styfmt{mfirstuc} (\sty{datatool-base} is loaded automatically with newer versions
of \sty{glossaries}). If available
\sty{mfirstuc} will now use \sty{datatool-base}'s
\ics{dtl@getfirst@UTFviii} command which is still experimental.
See the \styfmt{datatool} manual for further details.

\begin{verbatim}
\documentclass{article}

\usepackage[T1]{fontenc}
\usepackage[utf8]{inputenc}

\usepackage{datatool-base}[2016/01/12]% v2.24+
\usepackage{mfirstuc}

\begin{document}
\makefirstuc{élite}
\end{document}
\end{verbatim}
(Package ordering is important.)

\begin{important}
In version 1.02 of \styfmt{mfirstuc}, a bug fix resulted in a change
in output if the first object is a control sequence followed by
an empty group. Prior to version 1.02, |\makefirstuc{\ae{}bc}|
produced \ae Bc. However as from version 1.02, it now produces
\AE bc.
\end{important}

Note also that
\begin{verbatim}
\newcommand{\abc}{abc}
\makefirstuc{\abc}
\end{verbatim}
produces: ABC. This is because the first object in the argument of
\cs{makefirstuc} is \cs{abc}, so it does |\MakeUppercase{\abc}|.
Whereas:
\begin{verbatim}
\newcommand{\abc}{abc}
\expandafter\makefirstuc\expandafter{\abc}
\end{verbatim}
produces: Abc. There is a short cut command which will do this:
\begin{definition}[\DescribeMacro{\xmakefirstuc}]
\cs{xmakefirstuc}\marg{stuff}
\end{definition}
This is equivalent to \cs{expandafter}\cs{makefirstuc}\cs{expandafter}\marg{stuff}. So
\begin{verbatim}
\newcommand{\abc}{abc}
\xmakefirstuc{\abc}
\end{verbatim}
produces: 
\newcommand{\abc}{abc}%
\xmakefirstuc{\abc}.

\begin{important}
\cs{xmakefirstuc} only performs one level expansion on the
\emph{first} object in its argument. It does not fully expand the entire
argument.
\end{important}

As from version 1.10, there is now a command that fully expands the
entire argument before applying \cs{makefirstuc}:
\begin{definition}[\DescribeMacro\emakefirstuc]
\cs{emakefirstuc}\marg{text}
\end{definition}

Examples:
\begin{verbatim}
\newcommand{\abc}{\xyz a}
\newcommand{\xyz}{xyz}
No expansion: \makefirstuc{\abc}.
First object one-level expansion: \xmakefirstuc{\abc}.
Fully expanded: \emakefirstuc{\abc}.
\end{verbatim}
produces: 
\renewcommand{\abc}{\xyz a}%
\newcommand{\xyz}{xyz}
No expansion: \makefirstuc{\abc}.
First object one-level expansion: \xmakefirstuc{\abc}.
Fully expanded: \emakefirstuc{\abc}.

If you use \styfmt{mfirstuc} without the \sty{glossaries} package, 
the standard \cs{MakeUppercase} command is used. If used with
\sty{glossaries}, \ics{MakeTextUppercase} (defined by the \sty{textcase}
package) is used instead. If you are using \styfmt{mfirstuc}
without the \styfmt{glossaries} package and want to use
\ics{MakeTextUppercase} instead, you can redefine
\begin{definition}[\DescribeMacro\glsmakefirstuc]
\cs{glsmakefirstuc}\marg{text}
\end{definition}
For example:
\begin{verbatim}
\renewcommand{\glsmakefirstuc}[1]{\MakeTextUppercase #1}
\end{verbatim}
Remember to also load \sty{textcase} (\styfmt{glossaries} loads this
automatically).

\chapter{Capitalise the First Letter of Each Word in a Phrase or Sentence (Title Case)}
\label{sec:capitalisewords}

New to mfirstuc v1.06:
\begin{definition}[\DescribeMacro{\capitalisewords}]
\cs{capitalisewords}\marg{text}
\end{definition}
This command applies \ics{makefirstuc} to each word in \meta{text}
where the space character is used as the word separator. Note that
it has to be a plain space character, not another form of space,
such as \verb|~| or \cs{space}. Note that no expansion is performed
on \meta{text}. See \sectionref{sec:nocap} for excluding
words (such as \qt{of}) from the case-changing.

The actual capitalisation of each word is done using
(new to version 2.03): 
\begin{definition}[\DescribeMacro\MFUcapword]
\cs{MFUcapword}\marg{word}
\end{definition}
This just does \cs{makefirstuc}\marg{word} by default, but
its behaviour is determined by the conditional:
\begin{definition}[\DescribeMacro\ifMFUhyphen]
\cs{ifMFUhyphen}
\end{definition}

If you want to title case each part of
a compound word containing hyphens, you can enable this using
\begin{definition}[\DescribeMacro\MFUhyphentrue]
\cs{MFUhyphentrue}
\end{definition}
or switch it back off again using:
\begin{definition}[\DescribeMacro\MFUhyphenfalse]
\cs{MFUhyphenfalse}
\end{definition}
Compare
\begin{verbatim}
\capitalisewords{server-side includes}
\end{verbatim}
which produces: 
\begin{display}
\capitalisewords{server-side includes}
\end{display}
with
\begin{verbatim}
\MFUhyphentrue
\capitalisewords{server-side includes}
\end{verbatim}
which produces: 
\begin{display}
\MFUhyphentrue
\capitalisewords{server-side includes}
\end{display}

Formatting for the entire phrase must go outside
\cs{capitalisewords} (unlike \cs{makefirstuc}). Compare:
\begin{verbatim}
\capitalisewords{\textbf{a sample phrase}}
\end{verbatim}
which produces:
\begin{display}
\capitalisewords{\textbf{a sample phrase}}
\end{display}
with:
\begin{verbatim}
\textbf{\capitalisewords{a sample phrase}}
\end{verbatim}
which produces:
\begin{display}
\textbf{\capitalisewords{a sample phrase}}
\end{display}

As from version 2.03, there is now a command for phrases that may
include a formatting command:
\begin{definition}
\cs{capitalisefmtwords}\marg{phrase}
\end{definition}
where \meta{phrase} may be just words (as with \cs{capitalisewords})
or may be entirely enclosed in a formatting command in the form
\begin{alltt}
\cs{capitalisefmtwords}\{\meta{cs}\marg{phrase}\}
\end{alltt}
or contain formatted sub-phrases 
\begin{alltt}
\cs{capitalisefmtwords}\{\meta{words} \meta{cs}\marg{sub-phrase} \meta{words}\}
\end{alltt}
Avoid scoped declarations.

\begin{important}
\cs{capitalisefmtwords} is only designed for phrases that contain
text-block commands with a single 
argument, which should be a word or sub-phrase. Anything
more complicated is likely to break. Instead, use the
starred form or \cs{capitalisewords}.
\end{important}

The starred form only permits a text-block command at the
start of the phrase.

Examples:
\begin{enumerate}
\item Phrase entirely enclosed in a formatting command:
\begin{verbatim}
\capitalisefmtwords{\textbf{a small book of rhyme}}
\end{verbatim}
produces:
\begin{display}
\capitalisefmtwords{\textbf{a small book of rhyme}}
\end{display}

\item Sub-phrase enclosed in a formatting command:
\begin{verbatim}
\capitalisefmtwords{a \textbf{small book} of rhyme}
\end{verbatim}
produces:
\begin{display}
\capitalisefmtwords{a \textbf{small book} of rhyme}
\end{display}

\item Nested text-block commands:
\begin{verbatim}
\capitalisefmtwords{\textbf{a \emph{small book}} of rhyme}
\end{verbatim}
produces:
\begin{display}
\capitalisefmtwords{\textbf{a \emph{small book}} of rhyme}
\end{display}

\item Indicating words that shouldn't have the case changed
(see \sectionref{sec:nocap}):
\begin{verbatim}
\MFUnocap{of}
\capitalisefmtwords{\textbf{a \emph{small book}} of rhyme}
\end{verbatim}
produces:
\begin{display}
\MFUnocap{of}
\capitalisefmtwords{\textbf{a \emph{small book}} of rhyme}
\end{display}

\item Starred form:
\begin{verbatim}
\MFUnocap{of}
\capitalisefmtwords*{\emph{a small book of rhyme}}
\end{verbatim}
produces:
\begin{display}
\MFUnocap{of}
\capitalisefmtwords*{\emph{a small book of rhyme}}
\end{display}

\item The starred form also works with just text (no text-block
command):
\begin{verbatim}
\MFUnocap{of}
\capitalisefmtwords*{a small book of rhyme}
\end{verbatim}
produces:
\begin{display}
\MFUnocap{of}
\capitalisefmtwords*{a small book of rhyme}
\end{display}

\end{enumerate}

If there is a text-block command within the argument of the
starred form, it's assumed to be at the start of the argument.
Unexpected results can occur if there are other commands.
For example
\begin{verbatim}
\MFUnocap{of}
\capitalisefmtwords*{\emph{a small} book \textbf{of rhyme}}
\end{verbatim}
produces:
\begin{display}
\MFUnocap{of}
\capitalisefmtwords*{\emph{a small} book \textbf{of rhyme}}
\end{display}
(In this case \verb|\textbf{of rhyme}| is considered a
single word.) Similarly if the text-block command occurs
in the middle of the argument:
\begin{verbatim}
\MFUnocap{of}
\capitalisefmtwords*{a \emph{very small} book of rhyme}
\end{verbatim}
produces:
\begin{display}
\MFUnocap{of}
\capitalisefmtwords*{a \emph{very small} book of rhyme}
\end{display}
(In this case \verb|\emph{very small}| is considered a
single word.)

Grouping causes interference:
\begin{verbatim}
\capitalisefmtwords{{a \emph{small book}} of rhyme}
\end{verbatim}
produces:
\begin{display}
\capitalisefmtwords{{a \emph{small book}} of rhyme}
\end{display}
As with all the commands described here, avoid declarations.
For example, the following fails:
\begin{verbatim}
\capitalisefmtwords{{\bfseries a \emph{small book}} of rhyme}
\end{verbatim}
produces:
\begin{display}
\capitalisefmtwords{{\bfseries a \emph{small book}} of rhyme}
\end{display}

Avoid complicated commands in the unstarred version. For example,
the following breaks:
\begin{verbatim}
\newcommand*{\swap}[2]{{#2}{#1}}
\capitalisefmtwords{a \swap{bo}{ok} of rhyme}
\end{verbatim}
However it works okay with the starred form and the simpler
\cs{capitalisewords}:
\begin{verbatim}
\newcommand*{\swap}[2]{{#2}{#1}}
\capitalisefmtwords*{a \swap{bo}{ok} of rhyme}

\capitalisewords{a \swap{bo}{ok} of rhyme}
\end{verbatim}
Produces:
\begin{display}
\newcommand*{\swap}[2]{{#2}{#1}}
\capitalisefmtwords*{a \swap{bo}{ok} of rhyme}

\capitalisewords{a \swap{bo}{ok} of rhyme}
\end{display}
Note that the case change is applied to the first argument.

\begin{definition}[\DescribeMacro{\xcapitalisewords}]
\cs{xcapitalisewords}\marg{text}
\end{definition}
This is a short cut for
\cs{expandafter}\cs{capitalisewords}\cs{expandafter}\marg{text}.

As from version 1.10, there is now a command that fully expands the
entire argument before applying \cs{capitalisewords}:
\begin{definition}[\DescribeMacro\ecapitalisewords]
\cs{ecapitalisewords}\marg{text}
\end{definition}

There are also similar shortcut commands for the version that allows
text-block commands:
\begin{definition}[\DescribeMacro{\xcapitalisefmtwords}]
\cs{xcapitalisefmtwords}\marg{text}
\end{definition}
The unstarred version is a short cut for
\cs{expandafter}\cs{capitalisefmtwords}\cs{expandafter}\marg{text}.
Similarly the starred version of \cs{xcapitalisefmtwords}
uses the starred version of \cs{capitalisefmtwords}.

For full expansion:
\begin{definition}[\DescribeMacro\ecapitalisefmtwords]
\cs{ecapitalisefmtwords}\marg{text}
\end{definition}
Take care with this as it may expand non-robust semantic commands
to replacement text that breaks the functioning of
\cs{capitalisefmtwords}. Use robust semantic commands
where possible. Again this has a starred version that uses
the starred form of \cs{capitalisefmtwords}.

Examples:
\begin{verbatim}
\newcommand{\abc}{\xyz\space four five}
\newcommand{\xyz}{one two three}
No expansion: \capitalisewords{\abc}.
First object one-level expansion: \xcapitalisewords{\abc}.
Fully expanded: \ecapitalisewords{\abc}.
\end{verbatim}
produces: 
\begin{display}
\renewcommand{\abc}{\xyz\space four five}%
\renewcommand{\xyz}{one two three}
No expansion: \capitalisewords{\abc}.
First object one-level expansion: \xcapitalisewords{\abc}.
Fully expanded: \ecapitalisewords{\abc}.
\end{display}

(Remember that the spaces need to be explicit. In the second case
above, using \cs{xcapitalisewords}, the space before \qt{four} has
been hidden within \cs{space} so it's not recognised as a word
boundary, but in the third case, \cs{space} has been expanded to an
actual space character.)

Examples:
\begin{enumerate}
\item 
\begin{verbatim}
\capitalisewords{a book of rhyme.}
\end{verbatim}
produces:
\capitalisewords{a book of rhyme.}

\item
\begin{verbatim}
\capitalisewords{a book\space of rhyme.}
\end{verbatim}
produces:
\capitalisewords{a book\space of rhyme.}

\item
\begin{verbatim}
\newcommand{\mytitle}{a book\space of rhyme.}
\capitalisewords{\mytitle}
\end{verbatim}
produces:
\newcommand{\mytitle}{a book\space of rhyme.}
\capitalisewords{\mytitle}
(No expansion is performed on \cs{mytitle}.) Compare with next example:

\item
\begin{verbatim}
\newcommand{\mytitle}{a book\space of rhyme.}
\xcapitalisewords{\mytitle}
\end{verbatim}
produces:
\xcapitalisewords{\mytitle}

However
\begin{verbatim}
\ecapitalisewords{\mytitle}
\end{verbatim}
produces: 
\ecapitalisewords{\mytitle}
(\cs{space} has been expanded to an actual space character.)

\item 
\begin{verbatim}
\newcommand*{\swap}[2]{{#2}{#1}}
\capitalisewords{a \swap{bo}{ok} of rhyme}

\ecapitalisewords{a \swap{bo}{ok} of rhyme}
\end{verbatim}
produces:
\begin{display}
\newcommand*{\swap}[2]{{#2}{#1}}
\capitalisewords{a \swap{bo}{ok} of rhyme}

\ecapitalisewords{a \swap{bo}{ok} of rhyme}
\end{display}
This is because the argument of \cs{ecapitalisewords} is
fully expanded before being passed to \cs{capitalisewords} so
that last example is equivalent to:
\begin{verbatim}
\capitalisewords{a {ok}{bo} of rhyme}
\end{verbatim}
\end{enumerate}

\section{PDF Bookmarks}
\label{sec:pdfbookmarks}

\begin{important}
If you are using \sty{hyperref} and want to use
\cs{capitalisewords}, \cs{capitalisefmtwords}
or \ics{makefirstuc} (or the expanded variants) 
in a section heading, the PDF bookmarks won't be able to use the command 
as it's not expandable, so you will get a warning that looks like:
\begin{verbatim}
Package hyperref Warning: Token not allowed in a PDF string 
(PDFDocEncoding):
(hyperref)                removing `\capitalisewords'
\end{verbatim}
\end{important}

If you want to provide an alternative for the PDF bookmark, you can
use \sty{hyperref}'s \ics{texorpdfstring} command. For example:
\begin{verbatim}
\chapter{\texorpdfstring
  {\capitalisewords{a book of rhyme}}% TeX
  {A Book of Rhyme}% PDF
}
\end{verbatim}
Alternatively, you can use \sty{hyperref}'s mechanism for
disabling commands within the bookmarks. For example:
\begin{verbatim}
\pdfstringdefDisableCommands{%
 \let\capitalisewords\@firstofone
}
\end{verbatim}

See the \sty{hyperref} manual for further details.

\section{Excluding Words From Case-Changing}
\label{sec:nocap}

As from v1.09, you can specify words which shouldn't be capitalised unless they
occur at the start of \meta{text} using:
\begin{definition}[\DescribeMacro\MFUnocap]
\cs{MFUnocap}\marg{word}
\end{definition}
This only has a local effect. The global version is:
\begin{definition}[\DescribeMacro\gMFUnocap]
\cs{gMFUnocap}\marg{word}
\end{definition}

For example:
\begin{verbatim}
\capitalisewords{the wind in the willows}

\MFUnocap{in}%
\MFUnocap{the}%

\capitalisewords{the wind in the willows}
\end{verbatim}
produces:
\begin{display}
\capitalisewords{the wind in the willows}

\MFUnocap{in}%
\MFUnocap{the}%

\capitalisewords{the wind in the willows}
\end{display}
The list of words that shouldn't be capitalised can be cleared using
\begin{definition}[\DescribeMacro\MFUclear]
\cs{MFUclear}
\end{definition}

You can also simply place an empty group in front of a word
if you don't want that specific instance to be capitalised.
For example:
\begin{verbatim}
\MFUclear
\capitalisewords{the {}wind in the willows}
\end{verbatim}
produces:
\begin{display}
\MFUclear
\capitalisewords{the {}wind in the willows}
\end{display}
This is also a useful way of protecting commands that
shouldn't be parsed. For example:
\begin{verbatim}
\capitalisewords{this is section {}\nameref{sec:nocap}.}
\end{verbatim}
produces
\begin{display}
\capitalisewords{this is section {}\nameref{sec:nocap}.}
\end{display}
(No case-changing is applied to \verb|\nameref{sec:nocap}|.
It just happens to already be in title case.)

The package \sty{mfirstuc-english} loads \styfmt{mfirstuc} and uses
\cs{MFUnocap} to add common English articles and conjunctions, such
as ``a'', ``an'', ``and'', ``but''. You may want to add other
words to this list, such as prepositions but, as there's some
dispute over whether prepositions should be capitalised, I~don't
intend to add them to this package.

If you want to write a similar package for another language, all you
need to do is create a file with the extension \texttt{.sty}
that starts with
\begin{verbatim}
\NeedsTeXFormat{LaTeX2e}
\end{verbatim}
The next line should identify the package. For example, if you have
called the file \texttt{mfirstuc-french.sty} then you need:
\begin{verbatim}
\ProvidesPackage{mfirstuc-french}
\end{verbatim}
It's a good idea to also add a version in the final optional
argument, for example:
\begin{verbatim}
\ProvidesPackage{mfirstuc-french}[2014/07/30 v1.0]
\end{verbatim}
Next load \styfmt{mfirstuc}:
\begin{verbatim}
\RequirePackage{mfirstuc}
\end{verbatim}
Now add all your \cs{MFUnocap} commands. For example:
\begin{verbatim}
\MFUnocap{de}
\end{verbatim}
At the end of the file add:
\begin{verbatim}
\endinput
\end{verbatim}

Put the file somewhere on \TeX's path, and now you can use this
package in your document. You might also consider 
\href{http://ctan.org/upload}{uploading it to CTAN} in case 
other users find it useful.

\chapter{UTF-8}
\label{sec:utf8}

The \cs{makefirstuc} command works by utilizing the fact that, in
most cases, \TeX\ doesn't require a regular argument to be enclosed
in braces if it only consists of a single token. (This is why you
can do, say, \verb|\frac12| instead of \verb|\frac{1}{2}| or
\verb|x^2| instead of \verb|x^{2}|, although some users
frown on this practice.)

A~simplistic version of the \cs{makefirstuc} command is:
\begin{verbatim}
\newcommand*{\FirstUC}[1]{\MakeUppercase #1}
\end{verbatim}
Here
\begin{verbatim}
\FirstUC{abc}
\end{verbatim}
is equivalent to
\begin{verbatim}
\MakeUppercase abc
\end{verbatim}
and since \cs{MakeUppercase} requires an argument, it grabs the
first token (the character ``a'' in this case) and uses that as the
argument so that the result is: Abc.

The \sty{glossaries} package needs to take into account the fact
that the text may be contained in the argument of a formatting
command, such as \cs{acronymfont}, so \cs{makefirstuc} has to be
more complicated than the trivial \cs{FirstUC} shown above, but at
its basic level, \cs{makefirstuc} uses this same method and is the
reason why, in most cases, you don't need to enclose the first
character in braces. So if 
\begin{alltt}
\cs{MakeUppercase} \meta{stuff}
\end{alltt}
works,
then
\begin{alltt}
\cs{makefirstuc}\marg{stuff} 
\end{alltt}
should also work and so should
\begin{alltt}
\cs{makefirstuc}\{\cs{foo}\marg{stuff}\}
\end{alltt}
but if 
\begin{alltt}
\cs{MakeUppercase} \meta{stuff}
\end{alltt}
doesn't work, then neither will
\begin{alltt}
\cs{makefirstuc}\marg{stuff} 
\end{alltt}
nor
\begin{alltt}
\cs{makefirstuc}\{\cs{foo}\marg{stuff}\}
\end{alltt}

Try the following document:
\begin{alltt}
\cs{documentclass}\{article\}

\cs{usepackage}[utf8]\{inputenc\}
\cs{usepackage}[T1]\{fontenc\}

\cs{begin}\{document\}

\cs{MakeUppercase} \~abc

\cs{end}\{document\}
\end{alltt}

This will result in the error:
\begin{verbatim}
! Argument of \UTFviii@two@octets has an extra }.
\end{verbatim}
This is why \verb|\makefirstuc{|\texttt{\~abc}\verb|}| won't work.
It will only work if the character \texttt{\~a} is placed inside a
group.

The reason for this error message is due to \TeX\ having been written before
Unicode was invented. Although \texttt{\~a} may look like a single
character in your text editor, from \TeX's point of view it's \emph{two} 
tokens. So
\begin{alltt}
\cs{MakeUppercase} \~abc
\end{alltt}
tries to apply \cs{MakeUppercase} to just the first octet of \~a.
This means that the second octet has been separated from the first octet,
which is the cause of the error. In this case the argument isn't a
single token, so the two tokens (the first and second octet of \~a)
must be grouped:
\begin{alltt}
\cs{MakeUppercase}\{\~a\}bc
\end{alltt}

Note that \XeTeX\ (and therefore \XeLaTeX) is a modern
implementation of \TeX\ designed to work with Unicode and therefore
doesn't suffer from this drawback. Now let's look at the \XeLaTeX\
equivalent of the above example:
\begin{alltt}
\cs{documentclass}\{article\}

\cs{usepackage}\{fontspec\}

\cs{begin}\{document\}

\cs{MakeUppercase} \~abc

\cs{end}\{document\}
\end{alltt}

This works correctly when compiled with \XeLaTeX. This means
that \cs{makefirstuc}\verb|{|\texttt{\~abc}\verb|}| will work
\emph{provided you use \XeLaTeX\ and the \sty{fontspec} package}.

Version 2.24 of \sty{datatool-base} added the
command \cs{dtl@getfirst@UTFviii} which attempts to grab both octets.  If this command
has been defined, \sty{mfirstuc} will use it when it tries to split
the first character from the rest of the word. See the
\href{http://mirrors.ctan.org/macros/latex/contrib/datatool/datatool-code.pdf}{\styfmt{datatool}
documented code} for further details.

\PrintIndex
\end{document}
