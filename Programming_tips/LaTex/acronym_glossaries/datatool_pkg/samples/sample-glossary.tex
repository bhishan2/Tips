\documentclass{article}

\usepackage{datagidx}

\usepackage[colorlinks]{hyperref}

 % Make a glossary database labelled 'glossary' with given title.

\newgidx[heading=\section]{glossary}{Glossary}

 % Make an acronym database labelled 'acronyms' with given title.
 % A different heading for the table of contents is supplied.

\newgidx[heading={\section[Acronyms]}]{acronyms}{List of Acronyms}

 % Make a notation database labelled 'notation' with given title
 % Make it an unnumbered section

\newgidx[heading=\section*]{notation}{Notation}

 % Define some terms for the glossary

\DTLgidxSetDefaultDB{glossary}

\newterm[description={sea mammal with flippers that eats
fish}]{seal}
\newterm[description={large seal}]{sea lion}

\newterm{bravo}
\newterm
  [%
    label=bravocry,
    description={cry of approval (pl.\ bravos)},
    parent=bravo
  ]
  {bravo}
\newterm
  [% options
    label=bravokiller,
    description={hired ruffian or killer (pl.\ bravoes)},
    plural=bravoes,
    parent=bravo
  ]
 {bravo}

\newterm
 [% options
   plural={glossaries},
 ]%
 {glossary}

\newterm
 [% options
   label=glossarycol,
   plural={glossaries},
   description={collection of glosses},
   parent=glossary
 ]
 {glossary}

\newterm
 [% options
   label=glossarylist,
   plural={glossaries},
   description={list of technical words},
   parent=glossary
 ]
 {glossary}

\newterm
  [% options
   % description contains commas so it must be grouped
   description={a list of individual pages or page ranges
    (e.g.\ 1,2,4,7--9)}
  ]
  {page list}

\newterm
 [%
   plural={indices},
   description={an alphabetical list of names or subjects with
   references to their location in the document (pl.\ indices or
   indexes)}
 ]
 {index}

 % define some acronyms

\DTLgidxSetDefaultDB{acronyms}

\newacro{html}{hyper-text markup language}
\newacro{css}{cascading style sheet}

% override default description on this one:
\newacro[description={eXtensible Markup Language}]{xml}{extensible markup language}

% define some notation

\DTLgidxSetDefaultDB{notation}

\newterm
 [%
   symbol={\ensuremath{\mathcal{S}}},
   description={a collection of distinct objects},
 ]%
 {set}

\newterm
 [%
   symbol={\ensuremath{\mathcal{U}}},
   description={the set containing everything}
 ]%
 {universal set}

\newterm
 [%
   symbol={\ensuremath{\emptyset}},
   description={the set with no elements}
 ]%
 {empty set}

\newterm
 [%
   symbol={\ensuremath{|\mathcal{S}|}},
   description={the number of elements in the set
\ensuremath{\mathcal{S}}}
 ]%
 {cardinality}

\title{Sample Document Using the datagidx Package}
\author{Nicola L. C. Talbot}

\begin{document}

\maketitle

Here's an acronym referenced using \verb|\acr|: \acr{html}.
And here
it is again: \acr{html}. If you're used to the \texttt{glossaries}
package, note the difference in using \verb|\gls|: \gls{html}.
And again (no difference): \gls{html}.

Here are some more entries:

\acr{xml} and \acr{css}.
Next use:
\acr{xml} and \acr{css}.
Full form:
\gls{xml} and \gls{css}.

Reset again. \glsresetall{acronyms}%
Start with a capital. \Acr{html}.
Next: \Acr{html}. Full: \Gls{html}.

Prefer capitals?
\renewcommand{\acronymfont}[1]{\MakeTextUppercase{#1}}%
\Acr{xml}. Next: \acr{xml}. Full: \gls{xml}.

Prefer small-caps?
\renewcommand{\acronymfont}[1]{\textsc{#1}}%
\Acr{css}. Next: \acr{css}. Full: \gls{css}.

A \gls{set} (denoted \glssym{set}) is a collection of objects.
The \gls{universal set} is the set of everything.
The \gls{empty set} contains no elements.
The \gls{cardinality} of a set (denoted \glssym{cardinality}) is the
number of elements in the set.

A \gls{glossarylist} is a useful
addition to any technical document,
although a \gls{glossarycol} can also simply be a collection of
glosses, which is another thing entirely. Some documents have
multiple \glspl{glossarylist}. Entries in a \gls{glossarylist} or 
\gls{index} have an associated \gls{page list}. 

A \gls{bravocry} is a cry of approval (plural \glspl{bravocry}) but
a
\gls{bravokiller} can also be a hired ruffian or killer (plural
\glspl{bravokiller}).

\newpage

Resetting all acronyms.\glsresetall{acronyms}

Here are the acronyms again:

\Acr{html}, \acr{xml} and \acr{css}.

Next use:
\Acr{html}, \acr{xml} and \acr{css}.

Full form:
\Gls{html}, \gls{xml} and \gls{css}.

\newpage

Provide your own link text: \glslink{[textbf]css}{style sheet}.

\newpage

\printterms
 [
   database=glossary,% 'glossary' database
   columns=1,% one column page layout
   postdesc=dot,% put a full stop after the description
   style=gloss,% use 'gloss' style (sub-entries numbered)
   namefont={\textbf},% put the name in bold
   namecase=firstuc,% make the first letter of the name upper case
   child=noname,% don't print sub item names
   location=hide% hide the location list
 ]

\printterms
 [
   database=acronyms,% 'acronyms' database
   postdesc=dot,% put a full stop after the description
   columns=1,% one column page layout
   namefont={\textbf},% put the name (i.e. the abbreviation) in bold
   namecase=uc,% make the name upper case
   style=align,% use the 'align' style
 ]

% Change the way the description is formatted (the argument to
% \DTLgidxFormatDesc will always be a control sequence so
% \xmakefirstuc is needed instead of \makefirstuc)
\renewcommand{\DTLgidxFormatDesc}[1]{\xmakefirstuc{#1}}

\printterms
 [
   database=notation,% 'notation' database
   postdesc=dot,% put a full stop after the description
   columns=1,% one page column layout
   namefont={\textbf},% make the name bold
   style=indexalign,
   namecase=firstuc,% make first letter of name a capital
   symboldesc={symbol desc},% put symbol to the left of description
   %symboldesc={desc (symbol)},% put symbol to the right of the description
   sort={},% don't sort
% If the following widths are 0 (default) they take up their natural
% amount of space. If the widths are greater than 0, those fields
% are put in a box of the given width.
   symbolwidth={2cm},% width taken up by symbol
   locationwidth={1cm},% width taken up by location list
 ]



\end{document}
