\documentclass{article}
\usepackage[utf8]{inputenc}
\usepackage[acronym]{glossaries}

\makeglossaries

\newglossaryentry{latex}
{
        name=latex,
        description={Is a mark up language specially suited for
scientific documents}
}

\newglossaryentry{maths}
{
        name=mathematics,
        description={Mathematics is what mathematicians do}
}

\newglossaryentry{formula}
{
        name=formula,
        description={A mathematical expression}
}

\newacronym{gcd}{GCD}{Greatest Common Divisor}

\newacronym{lcm}{LCM}{Least Common Multiple}

\begin{document}

The \Gls{latex} typesetting markup language is specially suitable
for documents that include \gls{maths}. \Glspl{formula} are
rendered properly an easily once one gets used to the commands.

Given a set of numbers, there are elementary methods to compute
its \acrlong{gcd}, which is abbreviated \acrshort{gcd}. This
process is similar to that used for the \acrfull{lcm}.


\clearpage

\printglossary[type=\acronymtype]

\printglossary

\end{document}
