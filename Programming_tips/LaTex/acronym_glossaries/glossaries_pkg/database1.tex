 % This is a sample database of glossary entries
 % Only those entries used in the document with \glslink, \gls,
 % \glspl, and uppercase variants will have entries in the
 % glossary. Note that the type key is not used, as the
 % glossary type can be specified in \loadglsentries

\newglossaryentry{array}{name=array,
description={A list of values identified by a numeric value}}

\newglossaryentry{binary}{name=binary,
description={Pertaining to numbers represented in base 2}}

\newglossaryentry{comment}{name=comment,
description={A remark that doesn't affect the meaning of the
code}}

\newglossaryentry{global}{name=global,
description={Something that maintains its state when it leaves
the current group}}

\newglossaryentry{local}{name=local,
description={Something that only maintains its state until
it leaves the group in which it was defined/changed}}

